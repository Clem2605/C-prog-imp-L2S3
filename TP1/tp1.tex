\documentclass{article}
%% packages
%%i%%%%%%%%%%%%%%%%%%%%%%%%%%%%%%%%%%%%%%%%%%%%%%%%%%%%%%%%%%%%%%%%%%%%%%%%%%%%%%%
\usepackage{amssymb}
\usepackage{amsmath}
\usepackage{amsthm}%% proof etc
\usepackage{algorithm}
\usepackage[noend]{algorithmic}


\usepackage{hyperref} %% inclusion after algorithm: mandatory
\usepackage{url}

\usepackage{boxedminipage}
\usepackage{graphicx}%%[draft] : do not embed figs/picts
\usepackage{float}
\usepackage{rotating}

\usepackage{xspace}
\usepackage{nicefrac}
%\usepackage{underscore}

\usepackage{comment}
\usepackage{soul}
\usepackage{verbatim}%%block comment

\usepackage{csquotes} % \textquote{}

%% writing in French!
\usepackage[T1]{fontenc}    % for accents
\usepackage[utf8]{inputenc} % for coding
%\usepackage[french]{babel} % cesures, spaces etc
%% Misc 
\usepackage{xparse}   % \NewDocumentCommand
\usepackage{pifont}
\usepackage{marvosym}
%\usepackage{fp}       % floating point calculations: cf macro GanttLine
%\usepackage{eurosym}  % \euro symbol
%\usepackage{bibtopic}

\usepackage[usenames,dvipsnames]{xcolor}
\usepackage{listings}


\usepackage{fullpage}

%% macros
%%i%%%%%%%%%%%%%%%%%%%%%%%%%%%%%%%%%%%%%%%%%%%%%%%%%%%%%%%%%%%%%%%%%%%%%%%%%%%%%%%
\newif\ifSLIDES
\SLIDESfalse

\input{.latex_location_macros.sty}

%% General macros 
\input{\wmysty/macros-comments.sty}
\input{\wmysty/macros-envs.sty}
\input{\wmysty/macros-math-letters-fontified.sty}
\input{\wmysty/macros-symbols.sty}
\input{\wmysty/macros-algorithms-code.sty}

%% Macros specific to one broad topic/theme 
\input{\wmysty/macros-SBL.sty}
\input{\wmysty/macros-wp-delaunay-voronoi.sty}
\input{\wmysty/macros-wp-space-filling-models.sty}
\input{\wmysty/macros-wp-conformational-analysis.sty}

%% Macros specific to one/a few papers (PhD project)
%\renewcommand{\tored}{\color{black}}
%\renewcommand{\toblue}{\color{black}}

\title{TP1}
\author{Jules}

%% Main
%%i%%%%%%%%%%%%%%%%%%%%%%%%%%%%%%%%%%%%%%%%%%%%%%%%%%%%%%%%%%%%%%%%%%%%%%%%%%%%%%%
\begin{document}
\maketitle

\section[]{Règle de lisibilité}

\subsection{Exo 1}
\lstinputlisting[language=C]{exo1.c}

\subsection{Exo 2}

Les erreurs suivantes ont été corrigées:
\begin{enumerate}
    \item Les includes doivent etre au debut du fichier
    \item Il ne peut y avoir deux fonctions main
    \item La premiere fonction ne retourne pas de float
    \item la variable \texttt{nombre} n'est jamais utilisée et doit donc
    etre supprimé
    \item chaque instruction se termine par \texttt{;}
    \item les commentaires doivent etre indiqués avec \texttt{//} ou \texttt{/* */}
\end{enumerate}

\lstinputlisting[language=C]{exo2.c}

\section[]{Identificateur - Variable - Type de données}

\subsection{Exo 1}

\begin{itemize}
    \item \texttt{float r,r1,r2;} est correct
    \item \texttt{real x,y,z;} est faux car \texttt{real} n'est 
    pas un type de base
    \item \texttt{int for,main;} est faux car main est déclaré 
    comme une variable ce qui va rentrer en concurrence avec la 
    fonction main que le fichier doit obligatoirement avoir
    \item \texttt{char rs-232} est faux. Le \texttt{-} n'est pas 
    valide.
    \item \texttt{double d1;d2;d3;} il manque le type de donné de 
    \texttt{d2} et \texttt{d3}. Une autre possibilité serait de remplacer 
    les deux premier \texttt{;} par des \texttt{,}.
    \item \texttt{long int \_\_3;} est correct car \texttt{\_} compte 
    comme un charactère alphabétique.
    \item \texttt{unsigned char c='c';} est correct.
    
\end{itemize}

\subsection{Exo 2}

On suppose une architecture 16 bits.

\begin{tabular}{c|c}
    compteur $[0,300]$ & \texttt{unsigned int compteur;}\\
    x,y $[-120,100]$ & \texttt{char x,y;}\\
    mesure $[-10,10^4]$ & \texttt{int mesure;}\\
    surface $[0.5,150075]$ & \texttt{float surface;}\\
    nb1 $[-1,1024]$ & \texttt{short int nb1;}\\
    nb2 $[0,7\times 10^5]$ & \texttt{long nb2;}\\
    trouve $\{\text{vrai},\text{faux}\}$ & \texttt{int trouve;}\\

\end{tabular}

Note : Un mot est l'unité spécifique de l'architecture de 
l'ordinateur. En architecture 16 bit un mot est une unité de 
16 bits.

\subsection{Exo 3}

\begin{enumerate}
    \item \texttt{char}
    \item \texttt{int}
    \item \texttt{unsigned char}
    \item \texttt{float}
    \item \texttt{unsigned int}
    \item \texttt{long double}
    \item \texttt{long int}
\end{enumerate}

\section{TP machine - Entrées sorties}

\subsection{Exo 1}

\lstinputlisting[language=C]{exo3-1.c}

\subsection{Exo 2}

\lstinputlisting[language=C]{exo3-2.c}

Note : \texttt{scanf} retourne le nombre de variables correctement 
lues.

\subsection{Exo 3}
(a)

\lstinputlisting[language=C]{exo3-3.c}

(b)

Il faut modifier la ligne 
\begin{verbatim}
    scanf("%d %d %d",&jour,&mois,&année);
\end{verbatim}

par 

\begin{verbatim}
    scanf("%d/%d/%d",&jour,&mois,&année);
\end{verbatim}

\subsection{Exo 4}

(a)

\lstinputlisting[language=C]{exo3-4.c}

(b)

Les résultats ne sont plus valide car les nombres 30 000 et 500 000
sont trop grands pour le type \texttt{int}.

(c)

Il suffit de remplacer les \texttt{int} par \texttt{long}.


\end{document}